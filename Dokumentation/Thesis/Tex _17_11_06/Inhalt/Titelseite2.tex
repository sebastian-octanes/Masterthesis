%Define our elements for our title page
\makeatletter

\renewcommand{\title}[1]{\gdef\@title{#1}}
\newcommand{\artderarbeit}[1]{\gdef\@artderarbeit{#1}}
\renewcommand{\author}[1]{\gdef\@author{#1}}
\newcommand\@matrikelnummer{\relax}
\newcommand{\matrikelnummer}[1]{\gdef\@matrikelnummer{#1}}
\newcommand{\professor}[1]{\gdef\@professor{#1}}
\newcommand\@studiengang{\relax}
\newcommand{\studiengang}[1]{\gdef\@studiengang{#1}}
\newcommand{\university}[1]{\gdef\@university{#1}}
\newcommand{\faulogo}[1]{\gdef\@faulogo{Abbildungen/#1}}
\newcommand{\likelogo}[1]{\gdef\@likelogo{Abbildungen/#1}}
\newcommand{\submitdate}[1]{\gdef\@submitdate{#1}}
\newcommand{\startdate}[1]{\gdef\@startdate{#1}}
\newcommand{\declaration}[1]{\gdef\@declaration{#1}}
\newcommand\@betreuerB{\relax}
\newcommand{\betreuerB}[1]{\gdef\@betreuerB{#1}}
\newcommand\@betreuerC{\relax}
\newcommand{\betreuerC}[1]{\gdef\@betreuerC{#1}}
\newcommand\@betreuerD{\relax}
\newcommand{\betreuerD}[1]{\gdef\@betreuerD{#1}}
 
%-----------------------------------------------------------------------
% Title page definition.
%-----------------------------------------------------------------------
\newcommand{\titlep}{
        \thispagestyle{empty}
        \begin{titlepage}
		\enlargethispage{4cm}
%        \null
		\vspace*{-2cm} 
        \begin{center}
{\rmfamily\large \textbf{ Friedrich-Alexander-Universität Erlangen-Nürnberg}}
\vskip 1,4cm
\begin{center}
%\includegraphics[scale=0.80, natwidth=1024, natheight=200]{\@faulogo} \hfill
\includegraphics[scale=0.80]{\@faulogo} \hfill
\vskip .8cm
{\large \bfseries Lehrstuhl für Informationstechnik\\(Schwerpunkt Kommunikationselektronik)}
\vskip 1.2cm
%\includegraphics[scale=0.40, natwidth=1024, natheight=200]{\@likelogo} \hfill
\includegraphics[scale=0.40]{\@likelogo} \hfill
\end{center}
\vskip .3cm
{\Large \@artderarbeit~mit dem Thema:\\}
\vskip 1.7cm
{\rmfamily\Large\bfseries\expandafter{\@title}}
        \end{center}
\vskip 1.5cm
		
\begin{table}[h!]
	\flushleft
%\setlength{\tabcolsep}{10pt}
	\large
\renewcommand{\arraystretch}{1,1}
	\begin{tabular}{p{4cm}l}
Bearbeiter     & {\rmfamily\large\expandafter{\@author}}\tabularnewline[.6ex]
\ifthenelse{\equal{\@matrikelnummer}{\relax}}{} {Matrikelnr. & \textrm{\@matrikelnummer}\tabularnewline[.6ex]}
\ifthenelse{\equal{\@studiengang}{\relax}}{}    {Studiengang & \textrm{\@studiengang}\tabularnewline[.6ex]}
Betreuer  & {\rmfamily\large\expandafter{\@professor}}\tabularnewline[.6ex]
                & \ifthenelse{\equal{\@betreuerB}{\relax}}{\vspace*{-0.6cm}}{\rmfamily\large{\@betreuerB}} \tabularnewline[.6ex]
                & \ifthenelse{\equal{\@betreuerC}{\relax}}{\vspace*{-0.9cm}}{\rmfamily\large{\@betreuerC}}  \tabularnewline[.6ex]
                & \ifthenelse{\equal{\@betreuerD}{\relax}}{\vspace*{-0.7cm}}{\rmfamily\large{\@betreuerD}} \tabularnewline[.6ex] Beginn         & {\rmfamily\large\expandafter{\@startdate}}\tabularnewline[.6ex]
Ende           & {\rmfamily\large\expandafter{\@submitdate}}            
	\end{tabular}
\end{table}
	        \vskip1cm
        \end{titlepage}
        \newpage}
        \makeatother
        
 


%-----------------------------------------------------------------------
% redefined \maketite
%-----------------------------------------------------------------------
\renewcommand{\maketitle}{%
                \titlep
	     }
%-----------------------------------------------------------------------
%  hier die eigenen Daten eintragen
%----------------------------------------------------------------------
    
\title{Latex für Abschlussarbeiten} % Mussfeld
\artderarbeit{Masterarbeit}         % Mussfeld
\author{Name Student}               % Mussfeld
\studiengang{Elektrotechnik-Elektronik-Informationstechnik}        % if Feld
\matrikelnummer{1234545646}         % if Feld
\faulogo{fau.png}                   % Mussfeld
\likelogo{like.png}                 % Mussfeld
\professor{Prof. Dr.-Ing. Jörn Thielecke}       % Mussfeld
\betreuerB{Markus Hiller, M.\,Sc.}           % if Feld
\betreuerC{Florian Particke, M.\,Sc.}              % if Feld
%\betreuerD{Betreuer D}             % if Feld  u.U. auskommentieren 
\startdate{01. Januar 2015}         % Mussfeld
\submitdate{08. August 2016}        % Mussfeld

% \maketitle

\maketitle