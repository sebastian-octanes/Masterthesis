%Thema und Aufgabenstellung
 
%\pagestyle{scrheadings}

\chapter*{Thema und Aufgabenstellung}
\markboth{Thema und Aufgabenstellung}{Thema und Aufgabenstellung}
\textbf{Thema:}\par\smallskip

 Modellfehler in optimierungsbasierter kombinierter Planung und Regelung für Rennwagen

\par\bigskip
\textbf{Aufgabenstellung:}\par\smallskip
\par\smallskip  
Die Automatisierung des Fahrens schließt sowohl die Planung als auch die Regelung des Fahrzeugs mit ein. Häufig werden beide Bestandteile hierarchisch voneinander getrennt. Dies ist sinnvoll, solange das kontrollierte Fahrzeug sicher innerhalb der Aktuatorlimitierungen betrieben werden soll, oder wenn die Trennung bereits durch die Problemstellung gegeben ist (Zieltrajektorie bereits vorgegeben) \cite{Williams2016AggressiveDW}.  

In anderen Fällen, z.B. wenn die gewünschte Dynamik wie in einer Rennsituation im Grenzbereich liegt, bietet sich eine kombinierte Planung und Regelung an. In diesem Beispiel würde die Kostenfunktion eine Minimierung der Rundenzeit beinhalten, während gleichzeitig die Beschränkungen des Fahrzeugs berücksichtigt werden.

Für derartige Probleme ist die modellprädiktive Regelung (MPC) bzw. eines ihrer Derivate besonders geeignet. Dabei kommt es immer zu einem sogenannten Modellfehler, der von der Komplexität und Genauigkeit des verwendeten Modells abhängt.

Das Ziel dieser Arbeit ist es, den Abfall bei der Leistung des Regelungsansatzes durch den Modellfehler zu untersuchen. Dafür soll eine Simulation verwendet werden.
Die Arbeit soll folgende Punkte beinhalten:
\begin{itemize}
\item Auswahl einer passenden Simulationsumgebung und deren Inbetriebnahme
\item Implementierung verschiedener (gegebener) Modelle für die Simulation
\item Implementierung des MPC-Ansatzes
\item Entwicklung einer einfachen Evaluationsmethode, um die Leistungsfähigkeit des Reglers zu untersuchen
\item Vergleich verschiedener Kombinationen aus Regler- und Simulationsmodellen
\end{itemize}