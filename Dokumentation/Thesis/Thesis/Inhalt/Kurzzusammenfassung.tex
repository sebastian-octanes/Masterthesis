%Kurzzusammenfassung

\chapter*{Kurzzusammenfassung}

Mit zunehmender Rechenleistung und Erfahrung der Automobilbranche mit autonomen Fahrzeugen rückt auch das Thema der selbstfahrenden Rennautos immer mehr in den Fokus. Das ROBORACE Projekt ist hier Vorreiter mit seiner ausgefeilten Hardwareplattform und dem bereits in öffentlichen Events gezeigten Fahrleistungen. Auch die Formula Student (FS) verschließt sich nicht diesem Trend und hat 2017 die Rubrik Driverless ins Leben gerufen. \\
Diese Masterarbeit beschreibt einen Ansatz zur Echtzeitregelung und Trajektionsplanung für ein eben solches Driverless-Racecar. Die Basis hierfür ist ein Model Predictive Control (MPC) Algorithmus. Er vereint die Regelung und Trajektionsplanung und ist sehr adaptiv bezüglich verschiedener Fahrsituationen und Ziele.
Als Ausgangssituation wird angenommen, dass das Fahrzeug bereits eine Runde auf einem unbekannten Kurs absolviert und nun eine genaue Karte des Kurses errechnet hat. 
Um das MPC nutzen zu können muss ein Fahrzeugmodell hinterlegt werden. Je genauer dieses ist, desto näher kann die Regelung an die Grenzen des realen Fahrzeuges gehen. Neben der Auslegung für das aktuellste FS-Fahrzeug des High Octane Motorsports für die Driverless Umrüstung, wird untersucht ab welchem Punkt ein kinematisches Modell nicht mehr ausreicht um das Fahrzeug sicher auf dem Rennkurs zu führen.


